\documentclass[aspectratio=169, 12pt]{beamer}
 
\usepackage[utf8]{inputenc}
\usepackage{natbib, url}
\usepackage{enumerate, amsmath, amssymb, amsthm}
\usepackage{ragged2e} % make it justified
\justifying

% \usepackage{enumitem}
% \setlist{itemsep = 0pt, topsep = 1pt, leftmargin = 0.6mm}

\usepackage{hyperref}
\hypersetup{colorlinks, citecolor=blue, urlcolor=blue}
\usepackage{booktabs}
\usepackage{graphicx}


%\mode<presentation>{}
%\usepackage{beamerthemesplit} 

\setbeamertemplate{footline}[frame number]
%\setbeamertemplate{headline}{}
 
 
%Information to be included in the title page:
\title{Statistical tests, P values, confidence intervals, and power: a guide
to misinterpretations}

\author{Jesse DiMarzo}
 
 
\begin{document}
 
\frame{\titlepage}

 
\section{Introduction}
\begin{frame}{Importance of Statistical Literacy}
\begin{itemize}
    \item Misinterpretations in statistical results remain widespread.
    \item Critical for improving scientific rigor and communication. 
    \item Paper discusses 25 common statistical misinterpretations.
\end{itemize}
\end{frame}


\begin{frame}{About the Authors}
\begin{itemize}
    \item \textbf{Sander Greenland}: Lead author, Professor of Epidemiology and Statistics, UCLA. Expert in statistical theory and epidemiologic methods.
    \item \textbf{Stephen J. Senn}: Competence Center for Methodology and Statistics, Luxembourg Institute of Health. Renowned for work in clinical trials and biostatistics.
    \item \textbf{John B. Carlin}: Clinical Epidemiology and Biostatistics Unit, University of Melbourne. Focus on public health and biostatistics.
    \item \textbf{Charles Poole}: Department of Epidemiology, UNC Chapel Hill. Expertise in epidemiologic methods.
    \item \textbf{Steven N. Goodman}: Stanford University, expert in evidence synthesis and Bayesian statistics.
    \item \textbf{Douglas G. Altman}: Centre for Statistics in Medicine, University of Oxford. A leader in statistical methods for medical research.
\end{itemize}
\end{frame}



\begin{frame}{The Contextual Importance of Statistics}
\begin{itemize}
    \item Misinterpretation of statistical results is widespread.
    \item Journals have taken drastic actions (e.g., banning significance testing).
    \item Focus on educating and teaching concepts like P-values and confidence intervals, and the assumptions surrounding their models.
\end{itemize}
\end{frame}

\begin{frame}{Common Misinterpretations of Single P-values} % 1-2
\begin{itemize}
    \item P-value is the probability that the hypothesis is true.
    \item The P-value is the probability that chance produced the observed association.
\end{itemize}
\end{frame}

\begin{frame}{Misunderstanding Significance Results} % 3-5
\begin{itemize}
    \item Significant P-value means that the test(null) hypothesis is false.
    \item Nonsignificant P-value means that the test hypothesis is true.
    \item Large P-value is evidence in favor of the test hypothesis.
\end{itemize}
\end{frame}

\begin{frame}{Signifiance and Null Hypotheses} % 6-8
\begin{itemize}
    \item A P-value greater than 0.05 means no effect was observed.
    \item Statistical significance indicates substantive importance.
    \item Lack of significance indicates a small effect size. 
\end{itemize}
\end{frame}

\begin{frame}{Understanding Percentages in Context} % 9 - 11
\begin{itemize}
    \item The P-value is the chance of the data ocurring if the test hypothesis is true.
    \item If you reject the test hypothesis because \( P \leq 0.05 \), the chance you are in error is 5\%.
    \item \( P = 0.05 \) and \( P \leq 0.05 \) mean the same thing.
\end{itemize}
\end{frame}

\begin{frame}{Final Misconceptions for P-values} % 12 - 14
\begin{itemize}
    \item P-values are properly reported as inequalities.
    \item Statistical significacnce is a property of the phenomenon being studied.
    \item One should always use two-sided P-values.
\end{itemize}
\end{frame}

\begin{frame}{Controversial Interpretations of P-values}
\begin{itemize}
    \item P-values have been argued to overstate evidence against test hypotheses when compared to likelihood ratios or Bayes factors.
    \item Many statisticians dispute the use of these quantities as gold standards, asserting that P-values gauge error rates in decisions.
    \item From a frequentist view, \( 1 - P \) can measure evidence against the model used to compute the P-value.
    \item Controversies highlight differences in philosophical approaches to statistical interpretation.
\end{itemize}
\end{frame}

\begin{frame}{} 
\begin{itemize}
    \item
    \item
\end{itemize}
\end{frame}

\begin{frame}{}
\begin{itemize}
    \item
    \item
\end{itemize}
\end{frame}

\begin{frame}{}
\begin{itemize}
    \item
    \item
\end{itemize}
\end{frame}

\begin{frame}{}
\begin{itemize}
    \item
    \item
\end{itemize}
\end{frame}

\begin{frame}{}
\begin{itemize}
    \item
    \item
\end{itemize}
\end{frame}

\begin{frame}{}
\begin{itemize}
    \item
    \item
\end{itemize}
\end{frame}

\begin{frame}{}
\begin{itemize}
    \item
    \item
\end{itemize}
\end{frame}

\begin{frame}{}
\begin{itemize}
    \item
    \item
\end{itemize}
\end{frame}

\begin{frame}{}
\begin{itemize}
    \item
    \item
\end{itemize}
\end{frame}

\begin{frame}{}
\begin{itemize}
    \item
    \item
\end{itemize}
\end{frame}

\begin{frame}{}
\begin{itemize}
    \item
    \item
\end{itemize}
\end{frame}

\begin{frame}{}
\begin{itemize}
    \item
    \item
\end{itemize}
\end{frame}

\begin{frame}{}
\begin{itemize}
    \item
    \item
\end{itemize}
\end{frame}

\begin{frame}{}
\begin{itemize}
    \item
    \item
\end{itemize}
\end{frame}


\end{document}